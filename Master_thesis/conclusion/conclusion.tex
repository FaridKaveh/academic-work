\section{Conclusion}

The entropy production rate of non-equilibrium systems has been the subject of growing interest in the Stochastic Thermodynamics literature. Theoretical results regarding the entropy production of non-Markov processes have been developed. However, to date no paradigmatic, exactly solvable non-Markov models have appeared in the literature. In order to develop such a model, we investigate the path probabilities and entropy production rate of some candidate systems. 

A path-space formulation for continuous-time, discreet-space Markov chains is developed and validated through recovering the fully time-dependent probability flow and entropy production rate of the telegraph process. To the author's knowledge, this is the first derivation of a closed-form expression for the entropy production of any processes in a path-space framework in the literature on entropy. Although the additional difficulty involved in path-space analysis make this framework inadvisable for dealing with simple Markov chains, the path-space formulation is crucial in the investigation of non-Markov processes where entropy production can arise from asymmetric waiting-time distributions. Hence, the validation of this path-space formulation is an important result for the literature. 

The path-space framework is applied to a coarse-grained Markov chain (the `Waiting Room')  whereby an analytic, closed-form result for the entropy production rate of a non-Markov process is derived. Such a paradigmatic, exactly-solvable non-Markov model represents a step forward for the Stochastic Thermodynamics literature. The entropy production rate for the Waiting Room system is compared with that of the corresponding Markov chain and found to exhibit a number of unexpected features. In particular, the two systems exhibit qualitatively different behaviours in the limits as the reverse rate of the process becomes large or small. The expressions obtained also confirm for these systems a well-established theoretical result regarding the upper bound on the entropy production rate of a coarse-grained process. Moreover, the entropy production of the coarse-grained and the granular systems are found to be due to distinct mechanisms. Hence, the process of coarse-graining is found to affect not only the quantity but also the nature of entropy production. 

We derive the path probabilities of another coarse-grained Markov chain (the `Unrequited Love' system). These probabilities are presented in terms of a generalised generating function . This problem serves well to illustrate the difficulties in analysing non-Markov paths even in relatively simple systems. The main sources of difficulty are the non-trivial dependence of `hidden' processes on observable quantities, and the scaling of pertubative corrections with the number of transitions along a path. 

Moving to a continuous-space setting, a perturbation theory is developed for the entropy production rate of diffusion processes with stochastic, time-dependent drift terms beginning from the relevant Onsanger-Machlup functional. Crucially, This perturbation theory applies in those cases where the process driving the drift term is hidden, i.e. when the observable process is non-Markov. This general perturbation theory is new for the literature. This theory is applied to an asymmetric Run-and-Tumble (RnT) particle and the leading order contribution to the entropy production rate of this process is derived in closed form. In principle, it allows the entropy production rate of a process to be calculated up to arbitrary order whenever its drift is small compared to its diffusion (meaning small $v^2/D$, where $v$ is the characteristic drift and $D$ is the diffusion constant). The entropy production result for the asymmetric RnT particle demonstrates how a system driven by reversible dynamics can give rise to entropy production due to the nature of the physical coupling involved. Inspired by this result, we propose the `Unexpected State' mechanism of entropy production. 

We conclude with a novel classification of non-Markov processes that is intended to identify unifying paradigms and guide future research. In this penultimate section the concrete problems considered previously are presented in an abstract setting more conducive to a Probability Theoretic approach. Likewise our approach to each problem is formulated as it would apply in a general setting. It is hoped that this discussion can begin to bridge the gap between the rapidly advancing theory of Stochastic Processes and the study of stochastic systems in Statistical Physics, in particular Stochastic Thermodynamics. 

In terms of immediate applicability to physical or biological systems, future work must focus on relating the mathematical results presented here back to the key physical concepts of heat, free energy, and work. Once these relationships have been established, experimental work is needed to verify these results in real systems. Future work concerned with the theory of entropy production can use the waiting room system as an anchor to derive the entropy production rate of more complex systems. The perturbation theory derived in Section $\ref{chapter:RnT}$ presents an interesting link with contemporary research in Stochastic Analysis, whereby the theory of rough paths may be invoked to explain how, and in what sense, smooth functions can be understood to approximate a Brownian sample path. 